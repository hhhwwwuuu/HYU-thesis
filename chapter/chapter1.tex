\pagenumbering{arabic}

\chapter{Introduction}
\label{chapter1}

\section{aaa}

A Software Product Line (SPL) consists of the family of products that share the common set of features to accomplish the specific market segments. The main objective of SPL is to develop the family of software \cite{t1, t4, t5}. The family of software from SPL provides the auspicious way for the development of the large variety of software systems by the reusability of common and variable features from the core assets. Core asset of SPL consists of all required features that indicate the specification and scope of the family of software \cite{t7, t8, t9}. SPL is providing the best improvement in software industry due to the fast development of family of products by the reusability of features from core assets \cite{t10, t13}. 

Industry uses SPL such as the automobile, mobile phone, IoT applications etc. for the development of the family of products to achieve market segments. Industry claims that SPL provides the auspicious way for better, faster and cheaper development of the large variety of software systems. SPL organizations have reported enhancement in the order-of-magnitude related to quality, cost and time to market \cite{tr14, t15}. All products that aim to develop from SPL are differed from each other according to the perspective of individual end-users or market requirements. To acquire the high quality, less development cost and time to market, SPL is based on the reusability of software components \cite{t16, t11}.

Reusable components of SPL are common and variable features. These features are used to develop the family of products. Common features are easy to manage due to the reusability in all products that develop from the domain of SPL \cite{t3, t6, t12}. Variability features create differentiation and diversification in products and are selected according to end user of requirements \cite{t2, t17}. Therefore, variable features are not part of every product and need to manage the relationship with other features during selection and rejection from actual product development. During product configurations selection or rejection of a feature may cause the relationship constraints and leads to invalid product \cite{t18, t19}. Therefore, variability management of SPL is the key challenge during the development of the family of software. To enable the high reusability of SPL features, variable features need to handle and manage in efficient way as SPL supports the high reusability of features \cite{t20, tt20}.

To manage the common and variable features of SPL, a tree structure known as Feature Model is used in literature \cite{tr21, t22, t23}. Feature modeling is a method for the documentation of variable features in an SPL, how the variation points affect each other and what are the rules for establishing the configuration of SPL. Feature model is a compact picture of complete SPL with constraints and relationships among features. Each feature in feature model represents the functionality of product. For the development of unique product of SPL, features are selected from the feature model. During the selection of features from feature model, parent-child relationships must be followed for valid product configuration. Predefined relationships of feature model are, i) alternative features, where one and only one feature is selected from a group of features, ii) optional features, may or may not be selected and iii) OR group, at least one feature must be selected among features from a group \cite{tt17, tt18}. As the feature model is the representation of SPL domain, therefore, all features need to be developed in advance without developing any running application at the domain level for the reusability in actual product configurations. 


For the product configuration from feature model, organizations invest the effort, cost, time etc. in advance to construct the features. \cite{t24, t25, t26, t27}. Figure ~\ref{SPLCost} \cite{tt26} shows the initial cost of SPL and single product development that indicates the SPL organizations invest initial development cost without any market benefits. Therefore, the adaptation of SPL for any organization is based on the total initial development cost to estimate their own budget to develop the family of products.  Therefore, considering initial budget, organizations estimate the total cost and then take the decision whether they should adopt the specific SPL or not \cite{t28, t29, t30}. The initial development cost of SPL can be estimated by finding the total number of products.

The total number of products is calculated from the feature model by using cardinality relationships (alternative, optional, Or group). Evaluating the total number of products is simple and easy from the single level of relationship (cardinality constraints) feature models. When these relationships has become complex i.e. nested relationship, it also become hard to calculate the total number of products due to chances of relationship constraint violations. Therefore, in complex feature models, where nested cardinality relationships become more complex, calculation the total number of products is a challenging task especially in large feature models where millions of products exist. Furthermore, in simple feature models, valid product configurations and development of product according to market segments are easy due to the simple relationship. However, in complex feature models, it is a challenging and complex task to evaluate each product to estimate the initial development cost and market benefits due to nested relationships. Furthermore, From a large number of product configurations, developers need to select the product configuration according the end user and market segments. The final product must meet relationship constraints and the end-user specifications for the valid product configuration. In literature, proposed approaches such as Indicator-Based Evolutionary Algorithm (IBEA) \cite{t38} and Non-dominated Sorting Genetic Algorithm (NSGA-II) are unable to provide the best solution that meets the 100\% correctness i.e. relationship constraints and satisfy end-user requirements from the large and complex feature models. Therefore, from the millions of products, it is a challenging and time-consuming process to select the best solution i.e. 100\% correctness of optimum configuration that satisfies the market segments. 

To overcome the challenges and problems discussed above, we propose two approaches, the first approach enables to find the total number of products to estimate the initial cost of SPL and evaluate all the valid possible configurations. The second approach enables to find optimum configuration selection that satisfies the market segments.

First, to calculate the total number of products from complex feature models, we propose Binary Pattern for Nested Cardinality Constraints (BPNCC) approach. BPNCC approach enables to counts the total number of products and generates binary patterns for each product from complex feature model. To calculate the total number of products and valid configurations, BPNCC effectively manages the variability of feature model by converting the cardinality relationships of each parent-child relationship (alternative, optional, OR) into the binary pattern (0s, 1s), where 1 indicates selection of feature and 0 represents the rejection of feature. Furthermore, BPNCC approach also determines the valid combinations of features according to nested cardinality constraints of feature model. In BPNCC approach, variability feature model of binary combinations for selected and non-selected features are based on cardinality constraints, such as alternative, optional, and OR group. The beauty of this approach is independent of any tool and also does not hide the internal information of selected and non-selected features. In Order to show the correctness and effectiveness, we have applied the proposed approach to small and large feature models.

Second, we propose Multi-Objective Optimum (MOO)-BPNCC approach to get the goal-base and minimized, maximized optimum solution for application development without any relationship constraint violation. MOO-BPNCC approach is an extension of BPNCC approach and further consists of two independent and alternative methods to optimize feature model: the first method is suitable to get the goal-base optimization, it applies objective functions on all configurations for optimum solutions. However, this method increases space and execution time on large feature models where millions of product configurations exist. The second method is suitable to get the minimized and maximized optimum solutions by removing optional features that have constant values;  0 for minimization and 1 for maximization of objective functions. For minimized or maximized optimum solutions, execution time and space of feature model can be reduced. In BPNCC approach, we compute all possible solutions without any constraint violations; therefore, there is no possibility to miss any valuable solution for optimum combinations. In this study, we have found the minimized optimum solutions based on four minimized objective functions. We evaluated the outcomes of the goal-based method and minimized, maximized method. Minimized or maximized method is giving the best performance in context of execution time and space. Furthermore, we have performed the experimental comparison of MOO-BPNCC approach with well-known optimization algorithm IBEA from literature and concluded that MOO-BPNCC approach performs better to find the minimized and maximized optimum solutions. To evaluate the effectiveness and efficiency, we have compared MOO-BPNCC with IBEA. MOO-BPNCC finds the 100\% correctness of optimum configurations, whereas, IBEA performed maximum 96\% correctness of optimization of feature model.

The main contributions of this dissertation are given as:
\begin{itemize}
	\item For establishing the domain of SPL, organizations need to estimate the initial cost of SPL. Initial cost of SPL can be estimated by using the total number of products. Organizations can find the total number of products without missing any product by using our proposed BPNCC approach. Furthermore, for product configurations, the adaptation of BPNCC approach is simple and effective for the organizations to find the all possible product configurations of feature model in the binary pattern. The binary pattern of SPL configurations enables to calculate the cost of each product. Before the development of any individual product, organizations can estimate the initial cost.
	
	\item By using MOO-BPNCC approach, organizations can find minimized or maximized optimum solutions without any relationship constraint violation for the product development according to market segments. MOO-BPNCC approach also enables to find the goal-base optimum solution that satisfies the end-user objective functions. 
	
\end{itemize}               

The remaining dissertation is organized as follows. Chapter II provides a brief background related to Software Product Line, Feature model and Optimization. Chapter III discusses related work. Chapter IV presents Product configurations of feature model with BPNCC proposed method and also presents the MOO-BPNCC for multi-objective optimization of feature model by using BPNCC. Chapter V provides the brief experimental comparison of MOO-BPNCC with existing optimization algorithms. In the last, Chapter VI concludes dissertation with a summary of the dissertation and future works.