
\setstretch{2.0}

전직대통령의 신분과 예우에 관하여는 법률로 정한다. 국가는 농업 및 어업을 보호·육성하기 위하여 농·어촌종합개발과 그 지원등 필요한 계획을 수립·시행하여야 한다.

대법원장과 대법관이 아닌 법관의 임기는 10년으로 하며, 법률이 정하는 바에 의하여 연임할 수 있다. 원장은 국회의 동의를 얻어 대통령이 임명하고, 그 임기는 4년으로 하며, 1차에 한하여 중임할 수 있다.

대통령후보자가 1인일 때에는 그 득표수가 선거권자 총수의 3분의 1 이상이 아니면 대통령으로 당선될 수 없다. 감사원은 세입·세출의 결산을 매년 검사하여 대통령과 차년도국회에 그 결과를 보고하여야 한다.

국민의 모든 자유와 권리는 국가안전보장·질서유지 또는 공공복리를 위하여 필요한 경우에 한하여 법률로써 제한할 수 있으며, 제한하는 경우에도 자유와 권리의 본질적인 내용을 침해할 수 없다.

대통령이 제1항의 기간내에 공포나 재의의 요구를 하지 아니한 때에도 그 법률안은 법률로서 확정된다. 법관은 탄핵 또는 금고 이상의 형의 선고에 의하지 아니하고는 파면되지 아니하며, 징계처분에 의하지 아니하고는 정직·감봉 기타 불리한 처분을 받지 아니한다.